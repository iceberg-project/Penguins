% updated in April 2002 by Antje Endemann
% Based on CVPR 07 and LNCS, with modifications by DAF, AZ and elle, 2008 and AA, 2010, and CC, 2011; TT, 2014; AAS, 2016

\documentclass[runningheads]{llncs}
\usepackage{graphicx}
\usepackage{amsmath,amssymb} % define this before the line numbering.
\usepackage{color}
\usepackage[width=122mm,left=12mm,paperwidth=146mm,height=193mm,top=12mm,paperheight=217mm]{geometry}
\begin{document}
% \renewcommand\thelinenumber{\color[rgb]{0.2,0.5,0.8}\normalfont\sffamily\scriptsize\arabic{linenumber}\color[rgb]{0,0,0}}
% \renewcommand\makeLineNumber {\hss\thelinenumber\ \hspace{6mm} \rlap{\hskip\textwidth\ \hspace{6.5mm}\thelinenumber}}
% \linenumbers
\pagestyle{headings}
\mainmatter

\title{A very long title with deep neural networks variants applied on five databases and solving many problems occuring there.} % Replace with your title

\titlerunning{A very long title}

\authorrunning{authors running}

\author{Authors}

%Please write out author names in full in the paper, i.e. full given and family names. 
%If any authors have names that can be parsed into FirstName LastName in multiple ways, please include the correct parsing, in a comment to the volume editors:
%\index{Lastnames, Firstnames}
%(Do not uncomment it, because you may introduce extra index items if you do that...)

\institute{Department,\\
	University\\
	\email{ \{author1,author2\}@univ.edu}
}


\maketitle

\begin{abstract}
The abstract should summarize the contents of the paper. LNCS guidelines
indicate it should be at least 70 and at most 150 words. It should be set in 9-point
font size and should be inset 1.0~cm from the right and left margins.
\dots
\keywords{We would like to encourage you to list your keywords within
the abstract section}
\end{abstract}


\section{Introduction}


This document serves as an example paper. It illustrates the format
we expect authors to follow when submitting the camera ready version to ECCV. 
When submitting:
\begin{enumerate}
\item Make sure to include any further style files and fonts you may have used.
\item You may use sub-directories.
\item Make sure to use relative paths for referencing files.
\item Make sure the source you submit compiles.
\end{enumerate}

\section{Paper formatting}

\subsection{Language}

All manuscripts must be in English.

\subsection{Paper length}
Papers submitted for review should be complete. 
The length should match that intended for final publication. 
Papers accepted for the conference will be allocated 14 pages (plus references) in the proceedings. 
Note that the allocated 14 pages do not include the references. The reason for this policy
is that we do not want authors to omit references for sake of space limitations.

Papers with more than 14 pages (excluding references) will be rejected. 
This includes papers where the margins and
formatting are deemed to have been significantly altered from those
laid down by this style guide. 


\subsection{Mathematics}

Please number all of your sections and displayed equations.  Again,
this makes reviewing more efficient, because reviewers can refer to a
line on a page.  Also, it is important for readers to be able to refer
to any particular equation.  Just because you didn't refer to it in
the text doesn't mean some future reader might not need to refer to
it.  It is cumbersome to have to use circumlocutions like ``the
equation second from the top of page 3 column 1''.  (Note that the
line numbering will not be present in the final copy, so is not an
alternative to equation numbers).  Some authors might benefit from
reading Mermin's description of how to write mathematics:
\url{www.pamitc.org/documents/mermin.pdf‎}.


\section{Manuscript Preparation}

This is an edited version of Springer LNCS instructions adapted
for ECCV 2016 final paper submission.
You are strongly encouraged to use \LaTeX2$_\varepsilon$ for the
preparation of your
camera-ready manuscript together with the corresponding Springer
class file \verb+llncs.cls+.

We would like to stress that the class/style files and the template
should not be manipulated and that the guidelines regarding font sizes
and format should be adhered to. This is to ensure that the end product
is as homogeneous as possible.

\subsection{Printing Area}
The printing area is $122  \; \mbox{mm} \times 193 \;
\mbox{mm}$.
The text should be justified to occupy the full line width,
so that the right margin is not ragged, with words hyphenated as
appropriate. Please fill pages so that the length of the text
is no less than 180~mm.

\subsection{Layout, Typeface, Font Sizes, and Numbering}
Use 10-point type for the name(s) of the author(s) and 9-point type for
the address(es) and the abstract. For the main text, please use 10-point
type and single-line spacing.
We recommend using Computer Modern Roman (CM) fonts, Times, or one
of the similar typefaces widely used in photo-typesetting.
(In these typefaces the letters have serifs, i.e., short endstrokes at
the head and the foot of letters.)
Italic type may be used to emphasize words in running text. Bold
type and underlining should be avoided.
With these sizes, the interline distance should be set so that some 45
lines occur on a full-text page.

\subsubsection{Headings.}

Headings should be capitalized
(i.e., nouns, verbs, and all other words
except articles, prepositions, and conjunctions should be set with an
initial capital) and should,
with the exception of the title, be aligned to the left.
Words joined by a hyphen are subject to a special rule. If the first
word can stand alone, the second word should be capitalized.
The font sizes
are given in Table~\ref{table:headings}.
\setlength{\tabcolsep}{4pt}
\begin{table}
\begin{center}
\caption{Font sizes of headings. Table captions should always be
positioned {\it above} the tables. The final sentence of a table
caption should end without a full stop}
\label{table:headings}
\begin{tabular}{lll}
\hline\noalign{\smallskip}
Heading level & Example & Font size and style\\
\noalign{\smallskip}
\hline
\noalign{\smallskip}
Title (centered)  & {\Large \bf Lecture Notes \dots} & 14 point, bold\\
1st-level heading & {\large \bf 1 Introduction} & 12 point, bold\\
2nd-level heading & {\bf 2.1 Printing Area} & 10 point, bold\\
3rd-level heading & {\bf Headings.} Text follows \dots & 10 point, bold
\\
4th-level heading & {\it Remark.} Text follows \dots & 10 point,
italic\\
\hline
\end{tabular}
\end{center}
\end{table}
\setlength{\tabcolsep}{1.4pt}

Here are
some examples of headings: ``Criteria to Disprove Context-Freeness of
Collage Languages'', ``On Correcting the Intrusion of Tracing
Non-deterministic Programs by Software'', ``A User-Friendly and
Extendable Data Distribution System'', ``Multi-flip Networks:
Parallelizing GenSAT'', ``Self-determinations of Man''.

\subsubsection{Lemmas, Propositions, and Theorems.}

The numbers accorded to lemmas, propositions, and theorems etc. should
appear in consecutive order, starting with the number 1, and not, for
example, with the number 11.

\subsection{Figures and Photographs}
\label{sect:figures}

Please produce your figures electronically and integrate
them into your text file. For \LaTeX\ users we recommend using package
\verb+graphicx+ or the style files \verb+psfig+ or \verb+epsf+.

Check that in line drawings, lines are not
interrupted and have constant width. Grids and details within the
figures must be clearly readable and may not be written one on top of
the other. Line drawings should have a resolution of at least 800 dpi
(preferably 1200 dpi).
For digital halftones 300 dpi is usually sufficient.
The lettering in figures should have a height of 2~mm (10-point type).
Figures should be scaled up or down accordingly.
Please do not use any absolute coordinates in figures.

Figures should be numbered and should have a caption which should
always be positioned {\it under} the figures, in contrast to the caption
belonging to a table, which should always appear {\it above} the table.
Please center the captions between the margins and set them in
9-point type
(Fig.~\ref{fig:example} shows an example).
The distance between text and figure should be about 8~mm, the
distance between figure and caption about 5~mm.
\begin{figure}
\centering
\includegraphics[height=6.5cm]{eijkel2}
\caption{One kernel at $x_s$ ({\it dotted kernel}) or two kernels at
$x_i$ and $x_j$ ({\it left and right}) lead to the same summed estimate
at $x_s$. This shows a figure consisting of different types of
lines. Elements of the figure described in the caption should be set in
italics,
in parentheses, as shown in this sample caption. The last
sentence of a figure caption should generally end without a full stop}
\label{fig:example}
\end{figure}

If possible (e.g. if you use \LaTeX) please define figures as floating
objects. \LaTeX\ users, please avoid using the location
parameter ``h'' for ``here''. If you have to insert a pagebreak before a
figure, please ensure that the previous page is completely filled.


\subsection{Formulas}

Displayed equations or formulas are centered and set on a separate
line (with an extra line or halfline space above and below). Displayed
expressions should be numbered for reference. The numbers should be
consecutive within the contribution,
with numbers enclosed in parentheses and set on the right margin.
For example,
\begin{align}
  \psi (u) & = \int_{0}^{T} \left[\frac{1}{2}
  \left(\Lambda_{0}^{-1} u,u\right) + N^{\ast} (-u)\right] dt \; \\
& = 0 ?
\end{align}

Please punctuate a displayed equation in the same way as ordinary
text but with a small space before the end punctuation.

\subsection{Footnotes}

The superscript numeral used to refer to a footnote appears in the text
either directly after the word to be discussed or, in relation to a
phrase or a sentence, following the punctuation sign (comma,
semicolon, or full stop). Footnotes should appear at the bottom of
the
normal text area, with a line of about 2~cm in \TeX\ and about 5~cm in
Word set
immediately above them.\footnote{The footnote numeral is set flush left
and the text follows with the usual word spacing. Second and subsequent
lines are indented. Footnotes should end with a full stop.}


\subsection{Program Code}

Program listings or program commands in the text are normally set in
typewriter font, e.g., CMTT10 or Courier.

\noindent
{\it Example of a Computer Program}
\begin{verbatim}
program Inflation (Output)
  {Assuming annual inflation rates of 7%, 8%, and 10%,...
   years};
   const
     MaxYears = 10;
   var
     Year: 0..MaxYears;
     Factor1, Factor2, Factor3: Real;
   begin
     Year := 0;
     Factor1 := 1.0; Factor2 := 1.0; Factor3 := 1.0;
     WriteLn('Year  7% 8% 10%'); WriteLn;
     repeat
       Year := Year + 1;
       Factor1 := Factor1 * 1.07;
       Factor2 := Factor2 * 1.08;
       Factor3 := Factor3 * 1.10;
       WriteLn(Year:5,Factor1:7:3,Factor2:7:3,Factor3:7:3)
     until Year = MaxYears
end.
\end{verbatim}
%
\noindent
{\small (Example from Jensen K., Wirth N. (1991) Pascal user manual and
report. Springer, New York)}



\subsection{Citations}

The list of references is headed ``References" and is not assigned a
number
in the decimal system of headings. The list should be set in small print
and placed at the end of your contribution, in front of the appendix,
if one exists.
Please do not insert a pagebreak before the list of references if the
page is not completely filled.
An example is given at the
end of this information sheet. For citations in the text please use
square brackets and consecutive numbers: \cite{Alpher02},
\cite{Alpher03}, \cite{Alpher04} \dots

\section{Conclusions}

The paper ends with a conclusion. 


\clearpage\mbox{}Page \thepage\ of the manuscript.
\clearpage\mbox{}Page \thepage\ of the manuscript.
\clearpage\mbox{}Page \thepage\ of the manuscript.
\clearpage\mbox{}Page \thepage\ of the manuscript.
\clearpage\mbox{}Page \thepage\ of the manuscript.
\clearpage\mbox{}Page \thepage\ of the manuscript.
\clearpage\mbox{}Page \thepage\ of the manuscript.
This is the last page of the manuscript.
\par\vfill\par
Now we have reached the maximum size of the ECCV 2016 submission (excluding references).
References should start immediately after the main text, but can continue on p.15 if needed.

\clearpage

\bibliographystyle{splncs03}
\bibliography{egbib}
\end{document}
